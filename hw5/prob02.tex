\documentclass[12pt]{article}
 
\usepackage[margin=1in]{geometry} 
\usepackage{amsmath,amsthm,amssymb}
\usepackage{bm} % for bold math symbols
\usepackage{amstext} % for \text macro
\usepackage{array}   % for \newcolumntype macro
\newcolumntype{L}{>{$}l<{$}} % math-mode version of "l" column type
\newcolumntype{C}{>{$}c<{$}} % math-mode version of "c" column type
\newcolumntype{R}{>{$}r<{$}} % math-mode version of "r" column type
 
\newcommand{\N}{\mathbb{N}}
\newcommand{\Z}{\mathbb{Z}}

\newenvironment{theorem}[2][Theorem]{\begin{trivlist}
\item[\hskip \labelsep {\bfseries #1}\hskip \labelsep {\bfseries #2.}]}{\end{trivlist}}
\newenvironment{lemma}[2][Lemma]{\begin{trivlist}
\item[\hskip \labelsep {\bfseries #1}\hskip \labelsep {\bfseries #2.}]}{\end{trivlist}}
\newenvironment{exercise}[2][Exercise]{\begin{trivlist}
\item[\hskip \labelsep {\bfseries #1}\hskip \labelsep {\bfseries #2.}]}{\end{trivlist}}
\newenvironment{reflection}[2][Reflection]{\begin{trivlist}
\item[\hskip \labelsep {\bfseries #1}\hskip \labelsep {\bfseries #2.}]}{\end{trivlist}}
\newenvironment{proposition}[2][Proposition]{\begin{trivlist}
\item[\hskip \labelsep {\bfseries #1}\hskip \labelsep {\bfseries #2.}]}{\end{trivlist}}
\newenvironment{corollary}[2][Corollary]{\begin{trivlist}
\item[\hskip \labelsep {\bfseries #1}\hskip \labelsep {\bfseries #2.}]}{\end{trivlist}}

\begin{document}



\noindent
\large\textbf{HW 5, Problem 02} \hfill \textbf{Gregory Linkowski} \\
\normalsize CS 450 / ECE 491 \hfill linkows2 \\
Michael Heath \hfill due 11/2/16 \\


\vspace{5mm}
Given the three data points $(-1,1), (0,0), (1,1),$ determine the interpolating polynomial of degree two:
\vspace{-2mm} \\

\begin{exercise}{1}
	... using the monomial basis
\end{exercise} \vspace{-10mm}
\begin{proof}[]
	\begin{align*}
		\begin{bmatrix}	1 & -1 & 1 \\ 1 & 0 & 0 \\ 1 & 1 & 1	\end{bmatrix}
			\begin{bmatrix}	x_1 \\ x_2 \\ x_3 \end{bmatrix}
			&= \begin{bmatrix} 1 \\ 0 \\ 1	\end{bmatrix} \\[10pt]
		\begin{cases}
			1 = x_1 - x_2 + x_3 \\
			0 = x_1 \\
			1 = x_1 + x_2 + x_3
		\end{cases}
			& \Rightarrow \begin{cases}
				x_2 = x_3 - 1 \\
				x_1 = 0 \\
				1 = x_3 - 1 + x_3
			\end{cases} \\[10pt]
		x = [0 \; 0 \; 1]^T \\
		p_{2}(t) = 0 + 0t + 1t^2 
	\end{align*}
	\textbf{Answer 1.} \vspace{-7mm} \\
		\[ \boxed{p_{2}(t) = t^2} \]
\end{proof}

\begin{exercise}{2}
... using the Lagrange basis
\end{exercise} \vspace{-10mm}
\begin{proof}[]
	\begin{align*}
		\begin{bmatrix}	1 & 0 & 0 \\ 0 & 1 & 0 \\ 0 & 0 & 1	\end{bmatrix}
		\begin{bmatrix}	x_1 \\ x_2 \\ x_3 \end{bmatrix}
		&= \begin{bmatrix} 1 \\ 0 \\ 1	\end{bmatrix} \\[10pt]
		\begin{cases}
		1 = x_1 \\
		0 = x_2 \\
		1 = x_3
		\end{cases} \\[10pt]
		l_j(t_i) = \begin{cases}	1 \quad \textnormal{if } i = j \\ 0 \quad \textnormal{if } i \neq j \end{cases}&, i,j = 1, ..., n \\[10pt]
		p_{2}(t) &= 1 l_1(t) + 0 l_2(t) + 1 l_3(t)\\
		p_2(t) &= 1\frac{(t-0)(t-1)}{(-1 - 0)(-1 - 1)} + 0 + 1\frac{(t+1)(t-0)}{((1+1)(1-0)}\\	
	\end{align*}
	\textbf{Answer 2.} \vspace{-7mm} \\
		\[ p_{2}(t) = \frac{t(t-1)}{2} + \frac{t(t+1)}{2} \boxed{= t^2} \]
\end{proof}

\pagebreak
\begin{exercise}{3}
... using the Newton basis
\end{exercise} \vspace{-10mm}
\begin{proof}[]
	\begin{align*}
		a_{ij} &= \pi_j(t_i) = \prod_{k=1}^{j-1}(t-t_k) \\[10pt]
		& \begin{bmatrix}
			1 & 0 & 0 \\
			1 & (t_2-t_1) & 0 \\
			1 & (t_3-t_1) & (t_3-t_1)(t_3-t_2)	\end{bmatrix}
		\begin{bmatrix}	x_1 \\ x_2 \\ x_3 \end{bmatrix}
		= \begin{bmatrix} 1 \\ 0 \\ 1	\end{bmatrix} \\[10pt]
		& \begin{cases}
			1 = x_1 \\
			0 = x_1 + x_2(0 + 1) \\
			1 = x_1 + x_2(1 + 1) + x_3(1+1)(1-0)
		\end{cases} 
		\Rightarrow \begin{cases} x_1 = 1 \\ x_2 = -1 \\ x_3 = 1 \end{cases}\\[10pt]
		p_{2}(t) &= x_1 + x_2(t - t_1) + x_3(t - t_1)(t - t_2) \\
		p_{2}(t) &= 1 + (-1)(t - (-1)) + 1 (t - (-1))(t - (0)) \\
		p_{2}(t) &= 1 - (t + 1) + (t + 1)t \\
	\end{align*}
	\textbf{Answer 3.} \vspace{-7mm} \\
		\[ p_{2}(t) = 1 - t - 1 + t^2 + t \boxed{= t^2} \]
\end{proof}

\end{document}