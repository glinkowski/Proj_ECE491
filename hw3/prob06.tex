\documentclass[12pt]{article}
 
\usepackage[margin=1in]{geometry} 
\usepackage{amsmath,amsthm,amssymb}
\usepackage{bm} % for bold math symbols
\usepackage{amstext} % for \text macro
\usepackage{array}   % for \newcolumntype macro
\newcolumntype{L}{>{$}l<{$}} % math-mode version of "l" column type
\newcolumntype{C}{>{$}c<{$}} % math-mode version of "c" column type
\newcolumntype{R}{>{$}r<{$}} % math-mode version of "r" column type
 
\newcommand{\N}{\mathbb{N}}
\newcommand{\Z}{\mathbb{Z}}

\newenvironment{theorem}[2][Theorem]{\begin{trivlist}
\item[\hskip \labelsep {\bfseries #1}\hskip \labelsep {\bfseries #2.}]}{\end{trivlist}}
\newenvironment{lemma}[2][Lemma]{\begin{trivlist}
\item[\hskip \labelsep {\bfseries #1}\hskip \labelsep {\bfseries #2.}]}{\end{trivlist}}
\newenvironment{exercise}[2][Exercise]{\begin{trivlist}
\item[\hskip \labelsep {\bfseries #1}\hskip \labelsep {\bfseries #2.}]}{\end{trivlist}}
\newenvironment{reflection}[2][Reflection]{\begin{trivlist}
\item[\hskip \labelsep {\bfseries #1}\hskip \labelsep {\bfseries #2.}]}{\end{trivlist}}
\newenvironment{proposition}[2][Proposition]{\begin{trivlist}
\item[\hskip \labelsep {\bfseries #1}\hskip \labelsep {\bfseries #2.}]}{\end{trivlist}}
\newenvironment{corollary}[2][Corollary]{\begin{trivlist}
\item[\hskip \labelsep {\bfseries #1}\hskip \labelsep {\bfseries #2.}]}{\end{trivlist}}

\begin{document}



\noindent
\large\textbf{HW 3, Problem 06} \hfill \textbf{Gregory Linkowski} \\
\normalsize CS 450 / ECE 491 \hfill linkows2 \\
Michael Heath \hfill due 10/5/16 \\



\begin{exercise}{1}
	What is the Newton iteration for computing the square root of a positive number $y$ (i.e. for solving the equation $f(x)=x^2 -  y=0$, given $y$)?
\end{exercise}
\begin{proof}[]
	\[ \textnormal{The equations:  }
	\begin{cases}
		f(x) &= x^2 - y \\
		f'(x) &= 2x
	\end{cases} \]
	The Newton iteration takes the form:
	\begin{align*}
		x_{k+1} &= x_k - \frac{f(x_k)}{f'(x_k)} \\
		&= x_k - \frac{x_k^2 - y}{2 x_k} \\
		&= \frac{2x_k}{2} - \frac{x_k}{2} + \frac{y}{2x_k}
	\end{align*}
	\[ \boxed{x_{k+1} = \frac{1}{2}\left( x_k + \frac{y}{x_k} \right)} \]
\end{proof}

\begin{exercise}{2}
	If we assume that the starting guess has an accuracy of 4 bits, how many iterations would be necessary to attain 24-bit accuracy? 53-bit accuracy?
\end{exercise}
\begin{proof}[]
	The roots of $f(x)$ are simple (assuming $y \neq 0$), and therefore the convergence rate is quadratic. Thus starting from 4 bits, how many iterations to achieve \dots \\
	\[\boxed{ \textnormal{\dots 24-bit accuracy?} \; \bm{3} \qquad
			\textnormal{\hfill \dots 53-bit accuracy?} \; \bm{4} } \]
	If $y = 0$, then convergence is superlinear and the number of iterations will be greater. Most likely \textbf{4} \& \textbf{5}, respectively.
\end{proof}



\end{document}