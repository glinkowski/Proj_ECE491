\documentclass[12pt]{article}
 
\usepackage[margin=1in]{geometry} 
\usepackage{amsmath,amsthm,amssymb}
\usepackage{bm} % for bold math symbols
\usepackage{amstext} % for \text macro
\usepackage{array}   % for \newcolumntype macro
\newcolumntype{L}{>{$}l<{$}} % math-mode version of "l" column type
\newcolumntype{C}{>{$}c<{$}} % math-mode version of "c" column type
\newcolumntype{R}{>{$}r<{$}} % math-mode version of "r" column type
 
\newcommand{\N}{\mathbb{N}}
\newcommand{\Z}{\mathbb{Z}}

\newenvironment{theorem}[2][Theorem]{\begin{trivlist}
\item[\hskip \labelsep {\bfseries #1}\hskip \labelsep {\bfseries #2.}]}{\end{trivlist}}
\newenvironment{lemma}[2][Lemma]{\begin{trivlist}
\item[\hskip \labelsep {\bfseries #1}\hskip \labelsep {\bfseries #2.}]}{\end{trivlist}}
\newenvironment{exercise}[2][Exercise]{\begin{trivlist}
\item[\hskip \labelsep {\bfseries #1}\hskip \labelsep {\bfseries #2.}]}{\end{trivlist}}
\newenvironment{reflection}[2][Reflection]{\begin{trivlist}
\item[\hskip \labelsep {\bfseries #1}\hskip \labelsep {\bfseries #2.}]}{\end{trivlist}}
\newenvironment{proposition}[2][Proposition]{\begin{trivlist}
\item[\hskip \labelsep {\bfseries #1}\hskip \labelsep {\bfseries #2.}]}{\end{trivlist}}
\newenvironment{corollary}[2][Corollary]{\begin{trivlist}
\item[\hskip \labelsep {\bfseries #1}\hskip \labelsep {\bfseries #2.}]}{\end{trivlist}}

\begin{document}



\noindent
\large\textbf{HW 7, Problem 04} \hfill \textbf{Gregory Linkowski} \\
\normalsize CS 450 / ECE 491 \hfill linkows2 \\
Michael Heath \hfill due 12/7/16 \\


\vspace{5mm}
Write a few sentences in response to each of the following regarding your solutions to the previous problem:
\vspace{-2mm} \\

\vspace{5mm}
\begin{exercise}{1}
	Explain why changing the time step resulted in the difference between your first and second plots.
\end{exercise}% \vspace{-2mm}
\begin{proof}[]
	\textbf{Answer 1.} \\% \vspace{-7mm} \\
	For the solution to remain stable, we want the domain of dependence of the difference scheme to contain the domain of dependence of the PDE. For $\delta x = 0.05$ and $\delta t = 0.0012$, the slope of the line formed is 0.024, while for $\delta t = 0.0013$ the slope is 0.026. The narrower region formed by the lines for $\delta t = 0.0013$ does not entirely contain the domain of dependence of the PDE, and is therefore unstable. 
\end{proof}

\vspace{5mm}
\begin{exercise}{2}
	Describe the differences between your third plot and the first two. Explain why these differences occur.
\end{exercise}% \vspace{-10mm}
\begin{proof}[]
\textbf{Answer 2.} \\% \vspace{-7mm} \\
	Backwards Euler doesn't have the size restrictions on $\delta x \; \& \; \delta t$, so remains stable even with the much larger step size of $\delta t = 0.005$. There appears to be some slight noise near the boundaries of $x = 0 \; \& \; x = 1$. I believe this is caused by fitting the system of equations to $u_i^{k+1}$. The solution probably doesn't exactly fit the boundaries at $u(t,0) = u(t,1) = 0$, but we force these conditions, resulting in the slight warping near the boundary.
\end{proof}

\vspace{5mm}
\begin{exercise}{3}
	Describe and explain any differences between your results using Crank-Nicolson (the fourth plot) and the previous parts
\end{exercise}% \vspace{-10mm}
\begin{proof}[]
\textbf{Answer 3.} \\% \vspace{-7mm} \\
	The warping near the boundary isn't apparent in the Crank-Nicolson plot, but there is some noise near $t = 0$ that fades as $t$ increases (most noticeable near the peak). I suspect this is because the approach considers values at two time steps at once, and this noise is smoothed out as we move further from the initial condition.
\end{proof}

\pagebreak
\vspace{5mm}
\begin{exercise}{4}
	Describe and explain the differences between your results using the Semidiscrete system (the fifth plot) and the previous parts.
\end{exercise}% \vspace{-10mm}
\begin{proof}[]
\textbf{Answer 4.} \\% \vspace{-7mm} \\
	The Method of Lines approach appears to reduce the warping and noise present in the previous two methods. This makes sense, as it defines continuous functions $u(t)$ at each $x_i$. When the continuous functions are solved, they still consider solutions along each neighboring spatial $u^k$, making the solution robust.
\end{proof}

\end{document}