\documentclass[12pt]{article}
 
\usepackage[margin=1in]{geometry} 
\usepackage{amsmath,amsthm,amssymb}
\usepackage{bm} % for bold math symbols
\usepackage{amstext} % for \text macro
\usepackage{array}   % for \newcolumntype macro
\newcolumntype{L}{>{$}l<{$}} % math-mode version of "l" column type
\newcolumntype{C}{>{$}c<{$}} % math-mode version of "c" column type
\newcolumntype{R}{>{$}r<{$}} % math-mode version of "r" column type
 
\newcommand{\N}{\mathbb{N}}
\newcommand{\Z}{\mathbb{Z}}

\newenvironment{theorem}[2][Theorem]{\begin{trivlist}
\item[\hskip \labelsep {\bfseries #1}\hskip \labelsep {\bfseries #2.}]}{\end{trivlist}}
\newenvironment{lemma}[2][Lemma]{\begin{trivlist}
\item[\hskip \labelsep {\bfseries #1}\hskip \labelsep {\bfseries #2.}]}{\end{trivlist}}
\newenvironment{exercise}[2][Exercise]{\begin{trivlist}
\item[\hskip \labelsep {\bfseries #1}\hskip \labelsep {\bfseries #2.}]}{\end{trivlist}}
\newenvironment{reflection}[2][Reflection]{\begin{trivlist}
\item[\hskip \labelsep {\bfseries #1}\hskip \labelsep {\bfseries #2.}]}{\end{trivlist}}
\newenvironment{proposition}[2][Proposition]{\begin{trivlist}
\item[\hskip \labelsep {\bfseries #1}\hskip \labelsep {\bfseries #2.}]}{\end{trivlist}}
\newenvironment{corollary}[2][Corollary]{\begin{trivlist}
\item[\hskip \labelsep {\bfseries #1}\hskip \labelsep {\bfseries #2.}]}{\end{trivlist}}

\begin{document}



\noindent
\large\textbf{HW 2, Problem 03} \hfill \textbf{Gregory Linkowski} \\
\normalsize CS 450 / ECE 491 \hfill linkows2 \\
Michael Heath \hfill due 9/21/16 \\



\begin{exercise}{1} 
	How many Householder transformations are required? \\ 
\end{exercise}

\begin{proof}[]
	Three transformations are required. \\
\end{proof}


\begin{exercise}{2} 
	What does the first column of $\bm{A}$ become as a result of applying the first Householder transformation? \\ 
\end{exercise}

\begin{proof}[]
	$\bm{H}_1 \bm{A} = 
	\begin{bmatrix}
		-2 & -5 & -15 \\
		0 & 0 & -\frac{4}{3} \\
		0 & 1 & \frac{11}{3} \\
		0 & 2 & \frac{32}{3} \\
	\end{bmatrix} $
	such that the first column is 
	$ \begin{bmatrix}
		-2 \\ 0 \\ 0 \\ 0 \\
	\end{bmatrix} $ \\
\end{proof}


\begin{exercise}{3} 
	What does the first column then become as a result of applying the second Householder transformation? \\ 
\end{exercise}

\begin{proof}[]
	The first column remains unchanged:  
	$ \begin{bmatrix}
	-2 \\ 0 \\ 0 \\ 0 \\
	\end{bmatrix} $ \\
\end{proof}


\begin{exercise}{4} 
	How many Givens rotations would be required to compute the QR factorization of $\bm{A}$? \\ 
\end{exercise}

\begin{proof}[]
	Givens rotations introduce zeros one at a time, requiring six rotations. \\
\end{proof}




\end{document}