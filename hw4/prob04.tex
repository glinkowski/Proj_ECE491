\documentclass[12pt]{article}
 
\usepackage[margin=1in]{geometry} 
\usepackage{amsmath,amsthm,amssymb}
\usepackage{bm} % for bold math symbols
\usepackage{amstext} % for \text macro
\usepackage{array}   % for \newcolumntype macro
\newcolumntype{L}{>{$}l<{$}} % math-mode version of "l" column type
\newcolumntype{C}{>{$}c<{$}} % math-mode version of "c" column type
\newcolumntype{R}{>{$}r<{$}} % math-mode version of "r" column type
 
\newcommand{\N}{\mathbb{N}}
\newcommand{\Z}{\mathbb{Z}}

\newenvironment{theorem}[2][Theorem]{\begin{trivlist}
\item[\hskip \labelsep {\bfseries #1}\hskip \labelsep {\bfseries #2.}]}{\end{trivlist}}
\newenvironment{lemma}[2][Lemma]{\begin{trivlist}
\item[\hskip \labelsep {\bfseries #1}\hskip \labelsep {\bfseries #2.}]}{\end{trivlist}}
\newenvironment{exercise}[2][Exercise]{\begin{trivlist}
\item[\hskip \labelsep {\bfseries #1}\hskip \labelsep {\bfseries #2.}]}{\end{trivlist}}
\newenvironment{reflection}[2][Reflection]{\begin{trivlist}
\item[\hskip \labelsep {\bfseries #1}\hskip \labelsep {\bfseries #2.}]}{\end{trivlist}}
\newenvironment{proposition}[2][Proposition]{\begin{trivlist}
\item[\hskip \labelsep {\bfseries #1}\hskip \labelsep {\bfseries #2.}]}{\end{trivlist}}
\newenvironment{corollary}[2][Corollary]{\begin{trivlist}
\item[\hskip \labelsep {\bfseries #1}\hskip \labelsep {\bfseries #2.}]}{\end{trivlist}}

\begin{document}



\noindent
\large\textbf{HW 4, Problem 04} \hfill \textbf{Gregory Linkowski} \\
\normalsize CS 450 / ECE 491 \hfill linkows2 \\
Michael Heath \hfill due 10/19/16 \\


\vspace{5mm}
Consider the function $f : \mathbb{R}^2 \rightarrow \mathbb{R}$ defined by
\[ f(x) = \frac{1}{2}\left( x_1^2 - x_2 \right) ^2 + \frac{1}{2} \left( 1 - x_1 \right) ^2 \] 
\vspace{-2mm} \\

\begin{exercise}{1}
	At what point does $f$ attain a minumum?
\end{exercise} \vspace{-10mm}
\begin{proof}[]
	\begin{align*}
		f(x) &= \frac{1}{2}\left( x_1^4 - 2 x_1^2 x_2 + x_2^2 \right) + \frac{1}{2}\left( 1 - 2 x_1 + x_1^2 \right) \\
		 &= \frac{1}{2} x_1^4 + \frac{1}{2} x_1^2 - x_1 - x_1^2 x_2 + \frac{1}{2} x_2^2 + \frac{1}{2} \\[15pt]
		 %
		\nabla f(x) = \bm{0} &= \begin{bmatrix} 2x_1^3 + x_1 - 1 - 2 x_1 x_2 \\ -x_1^2 + x_2 \end{bmatrix} \\
		 &\Rightarrow \begin{cases} 0 = 2 x_1^3 + x_1 - 1 - 2 x^3 \\ x_2 = x_1^2 \end{cases} \\
		 &\Rightarrow \begin{cases} x_1 = 1 \\ x_2 = x_1^2 \end{cases}
	\end{align*}
	\textbf{Answer 1.} \vspace{-7mm} \\
	\begin{quote}
		The function $f(x)$ has a minimum at $(x,y) = (1,1)$.
	\end{quote}
\end{proof}

\pagebreak
\begin{exercise}{2}
	Perform one iteration of Newton's method for minimizing $f$ using as starting point $x_0 = [2 \; 2]^T$.
\end{exercise} \vspace{-10mm}
\begin{proof}[]
	\begin{align*}
		H_f(x) &= \begin{bmatrix} 6x_1^2 + 1 - 2x_2 & -2x_1 \\ -2x_1 & 1 \end{bmatrix} \quad \Rightarrow \quad
		H_f(x_0) = \begin{bmatrix} 21 & -4 \\ -4 & 1 \end{bmatrix} \\[15pt]
		%
		\textnormal{Newton iteration: } &\begin{cases} H_f(x_k) s_k = - \nabla f(x_k) \\ x_{k+1} = x_k + s_k \end{cases} \\[15pt]
		%
		H_f(x_0) s = - \nabla f(x_0) &\Rightarrow \begin{cases} -9 = 21s_1 - 4s_2 \\ 2 = -4s_1 + s_2 \end{cases} \\
		&\Rightarrow \begin{cases} -9 = 21s_1 - 16s_1 - 8 \\ s_2 = 4s_1 + 2 \end{cases}
			\Rightarrow \begin{cases} s_1 = -\frac{1}{5} &= -0.2 \\ s_2 = 4s_1 + 2 &= 1.2 \end{cases} \\[15pt]
		%
		x_{1} = x_0 + s &\Rightarrow \begin{bmatrix} 1.8 \\ 3.2 \end{bmatrix}
	\end{align*}
	\textbf{Answer 2.} \vspace{-7mm} \\
	\begin{quote}
		$x_1 = [1.8 \; \; 3.2]^T = [\frac{9}{5} \; \frac{16}{5}]^T$
	\end{quote}
\end{proof}

%\pagebreak
\begin{exercise}{3} In what sense is this a good step?
\end{exercise} \vspace{-10mm}
\begin{proof}[] \textnormal{ } \\
	\textbf{Answer 3.} \vspace{-7mm} \\
	\begin{quote}
		The first iteration brings the first entry of $x_k$ closer to the expected value of 1 (moving in the right \textit{direction}), while the second entry is farther from the expected value. \\
		Note that the method does converge within 4 steps.
	\end{quote}
\end{proof}

%\pagebreak
\begin{exercise}{4} In what sense is this a bad step?
\end{exercise} \vspace{-10mm}
\begin{proof}[] \textnormal{ } \\
	\textbf{Answer 4.} \vspace{-7mm} \\
	\begin{quote}
		After the first iteration, the second entry of $x_k$ is farther from the expected value of 1 (moving in the wrong \textit{direction}). It also moves farther in the wrong direction than the amount by which the first entry converges.
	\end{quote}
\end{proof}


\end{document}