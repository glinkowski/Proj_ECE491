\documentclass[12pt]{article}
 
\usepackage[margin=1in]{geometry} 
\usepackage{amsmath,amsthm,amssymb}
\usepackage{bm} % for bold math symbols
\usepackage{amstext} % for \text macro
\usepackage{array}   % for \newcolumntype macro
\newcolumntype{L}{>{$}l<{$}} % math-mode version of "l" column type
\newcolumntype{C}{>{$}c<{$}} % math-mode version of "c" column type
\newcolumntype{R}{>{$}r<{$}} % math-mode version of "r" column type
 
\newcommand{\N}{\mathbb{N}}
\newcommand{\Z}{\mathbb{Z}}

\newenvironment{theorem}[2][Theorem]{\begin{trivlist}
\item[\hskip \labelsep {\bfseries #1}\hskip \labelsep {\bfseries #2.}]}{\end{trivlist}}
\newenvironment{lemma}[2][Lemma]{\begin{trivlist}
\item[\hskip \labelsep {\bfseries #1}\hskip \labelsep {\bfseries #2.}]}{\end{trivlist}}
\newenvironment{exercise}[2][Exercise]{\begin{trivlist}
\item[\hskip \labelsep {\bfseries #1}\hskip \labelsep {\bfseries #2.}]}{\end{trivlist}}
\newenvironment{reflection}[2][Reflection]{\begin{trivlist}
\item[\hskip \labelsep {\bfseries #1}\hskip \labelsep {\bfseries #2.}]}{\end{trivlist}}
\newenvironment{proposition}[2][Proposition]{\begin{trivlist}
\item[\hskip \labelsep {\bfseries #1}\hskip \labelsep {\bfseries #2.}]}{\end{trivlist}}
\newenvironment{corollary}[2][Corollary]{\begin{trivlist}
\item[\hskip \labelsep {\bfseries #1}\hskip \labelsep {\bfseries #2.}]}{\end{trivlist}}

\begin{document}



\noindent
\large\textbf{HW 1, Problem 08} \hfill \textbf{Gregory Linkowski} \\
\normalsize CS 450 / ECE 491 \hfill linkows2 \\
Michael Heath \hfill due 9/7/16 \\



\begin{exercise}{1a} 
	Show that $\bm{A} = 
	\begin{bmatrix}
		0.1 & 0.2 & 0.3 \\
		0.4 & 0.5 & 0.6 \\
		0.7 & 0.8 & 0.9 \\
	\end{bmatrix} $ is singular. ...
\end{exercise}

\begin{proof}[]
	\begin{align*}
		A_{3i} &= 2 \times A_{2i} - A_{1i} \\
		[0.7\quad 0.8\quad 0.9] &= 2[0.4\quad 0.5\quad 0.6] - [0.1\quad 0.2\quad 0.3] \\
			&= [0.8\quad 1.0\quad 1.2] - [0.1\quad 0.2\quad 0.3] \\
			&= [0.7\quad 0.8\quad 0.9] \\
	\end{align*}
	\quote{Therefore, $rank(\bm{A_{3\times3}}) = 2 \neq n$ and $\bm{A}$ is singular.}
\end{proof}

\begin{exercise}{1b}
	Describe the set of solutions to the system $\bm{Ax}=\bm{b}$ if $\bm{b} = 
	\begin{bmatrix}
		0.1 \\
		0.3 \\
		0.5 \\
	\end{bmatrix} $
\end{exercise} 

\begin{proof}[]
	\begin{align*}
	\bm{x} =
	\begin{bmatrix}
		x_1 \\
		x_2 \\
		\frac{1}{2} - \frac{1}{2}x_2 \\
	\end{bmatrix}
	\end{align*}
\end{proof}


\begin{exercise}{2}
	If we were to use Gaussian elimination with partial pivoting to solve this system using exact arithmetic, at what point would the process fail?
\end{exercise}

\begin{proof}[]
	In applying Gaussian elimination, there is a point where the matrix looks like: \\
	\begin{align*}
	\begin{bmatrix}
		0.1 & 0.2 & 0.3 & | & 0.1 \\
		0.0 & -0.3 & -0.6 & | & -0.1 \\
		0.0 & -0.6 & -1.2 & | & -0.2 \\
	\end{bmatrix}  \\
	\end{align*}
	One option is to solve $Row_3 = 2 \times Row_2 - Row_3$, resulting in zeros for all entries in Row 3. Clearly at this point, any attempt at back-substitution would fail.
\end{proof}







\end{document}