\documentclass[12pt]{article}
 
\usepackage[margin=1in]{geometry} 
\usepackage{amsmath,amsthm,amssymb}
\usepackage{bm} % for bold math symbols
\usepackage{amstext} % for \text macro
\usepackage{array}   % for \newcolumntype macro
\newcolumntype{L}{>{$}l<{$}} % math-mode version of "l" column type
\newcolumntype{C}{>{$}c<{$}} % math-mode version of "c" column type
\newcolumntype{R}{>{$}r<{$}} % math-mode version of "r" column type
 
\newcommand{\N}{\mathbb{N}}
\newcommand{\Z}{\mathbb{Z}}

\newenvironment{theorem}[2][Theorem]{\begin{trivlist}
\item[\hskip \labelsep {\bfseries #1}\hskip \labelsep {\bfseries #2.}]}{\end{trivlist}}
\newenvironment{lemma}[2][Lemma]{\begin{trivlist}
\item[\hskip \labelsep {\bfseries #1}\hskip \labelsep {\bfseries #2.}]}{\end{trivlist}}
\newenvironment{exercise}[2][Exercise]{\begin{trivlist}
\item[\hskip \labelsep {\bfseries #1}\hskip \labelsep {\bfseries #2.}]}{\end{trivlist}}
\newenvironment{reflection}[2][Reflection]{\begin{trivlist}
\item[\hskip \labelsep {\bfseries #1}\hskip \labelsep {\bfseries #2.}]}{\end{trivlist}}
\newenvironment{proposition}[2][Proposition]{\begin{trivlist}
\item[\hskip \labelsep {\bfseries #1}\hskip \labelsep {\bfseries #2.}]}{\end{trivlist}}
\newenvironment{corollary}[2][Corollary]{\begin{trivlist}
\item[\hskip \labelsep {\bfseries #1}\hskip \labelsep {\bfseries #2.}]}{\end{trivlist}}

\begin{document}



\noindent
\large\textbf{HW 1, Problem 07} \hfill \textbf{Gregory Linkowski} \\
\normalsize CS 450 / ECE 491 \hfill linkows2 \\
Michael Heath \hfill due 9/7/16 \\



\begin{exercise}{1} 
	What is the determinant of $\bm{A} = 
	\begin{bmatrix}
		1 & 1 + \epsilon \\
		1 - \epsilon & 1\\
	\end{bmatrix} $ ?
\end{exercise}

\begin{proof}[]
	\begin{align*}
		det(\bm{A}) &= (1 \times 1) - (1-\epsilon)(1+\epsilon) \\
			&= 1 - (1 - \epsilon + \epsilon - \epsilon^2) \\
			&= \epsilon^2
	\end{align*}
\end{proof}


\begin{exercise}{2} 
	In floating-point arithmetic, for what range of values of $\epsilon$ will the computed value of the determinant be zero?
\end{exercise}

\begin{proof}[]
	\quote{First, (assuming rounding to nearest) if \\
		$|\epsilon| \leq \frac{1}{2}\epsilon_{mach}$, then $\bm{A} = 
		\begin{bmatrix}
		1 & 1 \\
		1 & 1 \\
		\end{bmatrix}$ and $det(\bm{A}) = 0$.} \\
	\vspace{2mm} \\
	Second, if there is no gradual underflow, such that UFL $\geq \epsilon_{mach}$, \\
	then $det(\bm{A}) = 0$ when (assuming rounding to nearest):
	\begin{align*}
		\epsilon^2 &\leq \frac{1}{2}\beta^L \\
		|\epsilon| &\leq \frac{1}{4}\sqrt{\beta^L} \\
	\end{align*}
	That is, if $|\epsilon^2|$ is less than half of the lowest positive number represented, then the answer will be rounded to zero. \\
\end{proof}


\begin{exercise}{3} 
	What is the LU factorization of $\bm{A}$?
\end{exercise}

\begin{proof}[]
	\begin{align*}
		\bm{MA} &= 
			\begin{bmatrix}
				1 & 0 \\
				m & 1 \\
			\end{bmatrix}
			\begin{bmatrix}
				1 & 1 + \epsilon \\
				1 - \epsilon & 1 \\
			\end{bmatrix} \\
		&= \begin{bmatrix}
				1 & 1 + \epsilon \\
				m + (1 - \epsilon) & m(1 + \epsilon) + 1 \\
			\end{bmatrix} \\
		&= \begin{bmatrix}
			1 & 1+\epsilon \\
			0 & \epsilon^2 \\
		\end{bmatrix} = \bm{U}
	\end{align*} \\
	\quote{Let $m = (-1 + \epsilon)$, then: 
		$ \bm{M} = \begin{bmatrix}
			1 & 0 \\
			(-1 + \epsilon) & 1 \\
		\end{bmatrix} $ }\\
	\begin{align*}
		\textnormal{and} \qquad \bm{A} &= \bm{LU} \\
		\bm{MA} &= \bm{U} \\
		\bm{L^{-1}A} &= \bm{L^{-1}LU} \\
		\bm{L} &= \bm{M^{-1}} \\
		\textnormal{thus} \qquad \bm{L} &= 
			\begin{bmatrix}
				1 & 0 \\
				1-\epsilon & 1 \\ 
			\end{bmatrix} \\
	\end{align*} \\
\end{proof}


\begin{exercise}{4} 
	In floating-point arithmetic, for what range of values of $\epsilon$ will the computed value of $\bm{U}$ be singular?
\end{exercise}

\begin{proof}[]
	Where $\bm{U} = 
		\begin{bmatrix}
			1 & 1+\epsilon \\
			0 & \epsilon^2 \\
		\end{bmatrix} $\\
	\quote{We want $det(\bm{U}) = \epsilon^2 = 0$. } \\
	\quote{Assuming a floating-point system with gradual underflow and rounding to nearest, then:} \\
	\begin{align*}
		\epsilon^2 &< \frac{1}{2}\epsilon_{mach} \\
		|\epsilon| &< \frac{1}{4}\sqrt{\epsilon_{mach}}
	\end{align*}
\end{proof}


\end{document}